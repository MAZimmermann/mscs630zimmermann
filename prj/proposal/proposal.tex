%%%%%%%%%%%%%%%%%%%%%%%%%%%%%%%%%%%%%%%%%
%
% MSCS 630
% Spring 2020
% Project Proposal
%
%%%%%%%%%%%%%%%%%%%%%%%%%%%%%%%%%%%%%%%%%

%----------------------------------------------------------------------------------------
%	PACKAGES AND OTHER DOCUMENT CONFIGURATIONS
%----------------------------------------------------------------------------------------

\documentclass[letterpaper, 10pt,DIV=13]{scrartcl} 

\usepackage[T1]{fontenc} % Use 8-bit encoding that has 256 glyphs
\usepackage[english]{babel} % English language/hyphenation
\usepackage{sectsty}
\usepackage{parskip}
\usepackage{lastpage}
\usepackage{setspace}
\usepackage{indentfirst}
\usepackage{titlesec}

\allsectionsfont{\normalfont\scshape} % Make all section titles in default font and small caps.

\usepackage{fancyhdr} % Custom headers and footers
\pagestyle{fancyplain} % Makes all pages in the document conform to the custom headers and footers

\fancyhead{} % No page header - if you want one, create it in the same way as the footers below
\fancyfoot[L]{} % Empty left footer
\fancyfoot[C]{} % Empty center footer
\fancyfoot[R]{page \thepage\ of \pageref{LastPage}} % Page numbering for right footer

%----------------------------------------------------------------------------------------
%	TITLE SECTION
%----------------------------------------------------------------------------------------

\newcommand{\horrule}[1]{\rule{\linewidth}{#1}} % Create horizontal rule command with 1 argument of height

\title{	
   \normalfont \normalsize
   \textsc{MSCS 630 - Spring 2020 - Dr. Rivas} \\ [10pt] % Header stuff.
   \horrule{0.5pt} \\[0.25cm] 	% Top horizontal rule
   \huge Project Proposal  \\     	    % Assignment title
   \horrule{0.5pt} \\[0.25cm] 	% Bottom horizontal rule
}

\author{Marcus A. Zimmermann \\ \normalsize Marcus.Zimmermann1@Marist.edu}

\date{\normalsize\today} 	% Today's date.

\begin{document}
\maketitle % Print the title

\section*{Healthcrypt}

\begin{spacing}{1.25}
I propose to build a web application - developed and deployed locally via virtualenv, flask, and other tools and libraries - for physicians, providing a personal repository for confidential patient information. After creating an account and logging into the application, the physician is presented with a homepage from which he or she can create, read, update, and delete patient records. These records are presented as a form, prompting the physician for the patient's age, height, weight, and other basic information. Encryption comes into play when the form is saved, and the physician returns to his or her homepage. Each entry in the form will be encrypted individually before being stored together, as a record in a database with multiple values. If the physician wishes to read and/or update the form after it's initial save in the database, each value in the record will have to be decrypted before being displayed in the form again. Lastly, the physicians' usernames and passwords will also be encrypted before being stored. This web application is not intended to protect information in transit. It simply provides secure storage for sensitive medical records.
\end{spacing}

\end{document}