%%%%%%%%%%%%%%%%%%%%%%%%%%%%%%%%%%%%%%%%%
%
% MSCS 630
% Spring 2020
% Project Milestone
%
%%%%%%%%%%%%%%%%%%%%%%%%%%%%%%%%%%%%%%%%%

%----------------------------------------------------------------------------------------
%	PACKAGES AND OTHER DOCUMENT CONFIGURATIONS
%----------------------------------------------------------------------------------------

\documentclass[letterpaper, 10pt,DIV=13]{scrartcl} 

\usepackage[T1]{fontenc} % Use 8-bit encoding that has 256 glyphs
\usepackage[english]{babel} % English language/hyphenation
\usepackage{amsmath,amsfonts,amsthm,xfrac} % Math packages
\usepackage{sectsty}
\usepackage{graphicx}
\usepackage[lined,linesnumbered,commentsnumbered]{algorithm2e}
\usepackage{listings}
\usepackage{enumitem}
\usepackage{parskip}
\usepackage{lastpage}
\usepackage{setspace}

\allsectionsfont{\normalfont\scshape} % Make all section titles in default font and small caps.

\usepackage{fancyhdr} % Custom headers and footers
\pagestyle{fancyplain} % Makes all pages in the document conform to the custom headers and footers

\fancyhead{} % No page header - if you want one, create it in the same way as the footers below
\fancyfoot[L]{} % Empty left footer
\fancyfoot[C]{} % Empty center footer
\fancyfoot[R]{page \thepage\ of \pageref{LastPage}} % Page numbering for right footer

\renewcommand{\headrulewidth}{0pt} % Remove header underlines
\renewcommand{\footrulewidth}{0pt} % Remove footer underlines
\setlength{\headheight}{13.6pt} % Customize the height of the header

\numberwithin{equation}{section} % Number equations within sections (i.e. 1.1, 1.2, 2.1, 2.2 instead of 1, 2, 3, 4)
\numberwithin{figure}{section} % Number figures within sections (i.e. 1.1, 1.2, 2.1, 2.2 instead of 1, 2, 3, 4)
\numberwithin{table}{section} % Number tables within sections (i.e. 1.1, 1.2, 2.1, 2.2 instead of 1, 2, 3, 4)

\binoppenalty=3000
\relpenalty=3000

%----------------------------------------------------------------------------------------
%	TITLE SECTION
%----------------------------------------------------------------------------------------

\newcommand{\horrule}[1]{\rule{\linewidth}{#1}} % Create horizontal rule command with 1 argument of height

\title{	
   \normalfont \normalsize
   \textsc{MSCS 630 - Spring 2020 - Dr. Rivas} \\ [10pt] % Header stuff.
   \horrule{0.5pt} \\[0.25cm] 	% Top horizontal rule
   \huge Project Milestone  \\     	    % Assignment title
   \horrule{0.5pt} \\[0.25cm] 	% Bottom horizontal rule
}

\author{Marcus A. Zimmermann \\ \normalsize Marcus.Zimmermann1@Marist.edu}

\date{\normalsize\today} 	% Today's date.

\begin{document}
\maketitle % Print the title

\section*{Abstract}
\begin{spacing}{1.25}
Healthcrypt provides physicians with a personal repository for confidential patient information. After creating an account and logging into the application, the physician is presented with a homepage from which he or she can create, read, update, and delete patient records. These records are presented as a form, prompting the physician for the patient's age, height, weight, and other basic information. Encryption comes into play when the form is saved, and the physician returns to his or her homepage. Each entry in the form will be encrypted individually before being stored together, as a record in a database with multiple values. If the physician wishes to read and/or update the form after it's initial save in the database, each value in the record will have to be decrypted before being displayed in the form again. Lastly, the physicians' usernames and passwords will also be encrypted before being stored. This web application is not intended to protect information in transit. It simply provides secure storage for sensitive medical records.
\end{spacing}

\section*{Introduction}
\begin{spacing}{1.25}
Describes the motivation of this work and outlines the rest of the paper.
\end{spacing}

\section*{Background}
\begin{spacing}{1.25}
Describes what other researchers in the same area have done, and how they perhaps could be improved.
\end{spacing}

\section*{Methodology}
\begin{spacing}{1.25}
Describes what is the approach taken in this paper.
\end{spacing}

\section*{Experiments}
\begin{spacing}{1.25}
Describes the experiments performed, including details on the data used.
\end{spacing}

\section*{Discussion}
\begin{spacing}{1.25}
Examines the results of the experiments and draws some conclusion about their significance.
\end{spacing}

\section*{Conclusion}
\begin{spacing}{1.25}
Summarizes the paper and its findings.
\end{spacing}

\section*{References}
\begin{spacing}{1.25}
Gives properly formatted references to other scholarly work that this work is built on. Note that the references should be scholarly, which means things like refereed conference and journal articles. Importantly, that rules out things like most websites, basic textbooks, and press articles.
\end{spacing}

\end{document}